\usepackage[utf8x]{inputenc}





\usepackage[pdftex]{graphicx} % Required for including pictures

\usepackage[pdftex,linkcolor=black,pdfborder={0 0 0}]{hyperref} % Format links for pdf


\usepackage[a4paper, lmargin=0.1666\paperwidth, rmargin=0.1666\paperwidth, tmargin=0.1111\paperheight, bmargin=0.1111\paperheight]{geometry} %margins
%\usepackage{parskip}

%\usepackage[all]{nowidow} % Tries to remove widows
\usepackage[protrusion=true,expansion=true]{microtype} % Improves typography, load after fontpackage is selected

\usepackage{lipsum} % Used for inserting dummy 'Lorem ipsum' text into the template
\usepackage{amsmath}
\usepackage{amsfonts}
\usepackage{braket}
\usepackage{subcaption}

\usepackage{tikz}

\usetikzlibrary{quantikz2}

\usepackage[
backend=biber,
style=alphabetic,
sorting=ynt
]{biblatex}
\addbibresource{sample.bib}

\usepackage{lmodern}
\usepackage{framed}
\usepackage{advdate}

%-----------------------
% Begin document
%----------------------



\title{ \begin{framed} Quantum Information Theory - 67749 \\ 
Exercise 1, \today \end{framed}  }
%\date{\today}
\date{\vspace{-5ex}}
\begin{document}

%\begin{framed}
\maketitle{ }    
%\end{framed}


\newcommand{\CCZ}{\textbf{CCZ}}
\newcommand{\CCX}{\textbf{CCX}}


\newcounter{enumcirc}
\setcounter{enumcirc}{1} 
\counterwithin{enumcirc}{section}


\newcommand{\advanceday}[1][11]{%
\begingroup
\AdvanceDate[#1]%
\today%
\endgroup
}%


\newcommand{\subqCircEx}[2]{\begin{subfigure}[t]{0.5\textwidth}
        \stepcounter{enumcirc} \caption*{ (\alph{enumcirc}) #1} \centering 
        #2
    \end{subfigure}
}

\newcommand{\qCircEx}[4]{\begin{figure*}[h!]
    \centering
    \subqCircEx{#1}{#2}
    ~ 
    \subqCircEx{#3}{#4}
\end{figure*}
}

\newcommand{\qCircExfullline}[2]{\begin{figure*}[h!]
    \stepcounter{enumcirc} \caption*{ (\alph{enumcirc}) #1}
        \centering 
        #2
\end{figure*}
}



\section{Submission Guidelines.}
\begin{itemize}
    \item Due date - \advanceday. 
    \item Make sure your submission is clear. Unreadable assignments will get zero score.  
    \item Using any Generative AI tool is forbidden but not enforced. Yet, please keep in mind that we might call you for an interview about your assignment. 
\end{itemize}

\newpage

\section{Tensors Products.}
\begin{enumerate}
    \item Show that $(A \otimes B)(C \otimes D) = (AC) \otimes (BD)$ whenever the dimensions are compatible. Use the ket-bra notation. Then show that if $U_1, U_2$ are unitaris then $U_1 \otimes U_2$ is also a unitary. 
    
    \item Let $v_A, v_B$ be the vectors of eigenvalues of the matrices $A,B$ respectively. Show that $v_A \otimes v_B$ is the vector of eigenvalues of $A \otimes B$.
    
\end{enumerate}


\section{Unambiguous Discrimination.}
Recall the distinguishing task, we are promised to be given either $\ket{\psi_{1}}$ or $\ket{\psi_{2}}$ and have to decide which it is. In the class, we partly proved that if a distinguisher can answer a third option, stands for "don't know", then the success probability is $ 1 - \cos\theta$. 

\begin{enumerate}
    \item \textbf{[ Not Mandatory ] } In your words. Point out exactly what we skipped in the proof, and repeat the success probability calculation from the lecture. You should get $ a\left( 1  -||\braket{\psi_{1}| \psi_{2}}||^2 \right)$. 
    \item Write $E_{3}$ explicitly in the $\{\ket{\psi_{1}}, \ket{\psi_{1}^\perp}\}$ basis. The answer should depend on $a$ and $\theta$.
 \item Show that $a = 1 + \cos \theta $ gives the optimal measurement.  
\end{enumerate}

\section{Density Matrices.}
\begin{enumerate}
\item Show that the density matrices of the mixed state resulting from picking uniformly random a quantum state from an orthonormal basis are always $\frac{1}{n}I$ regardless of the basis.  

\item Let $\rho$ be the density matrix of the \textbf{EPR}. Divide the qubits into two local systems $A$ and $B$.
    
    Compute the mixed state over $A$ given by first applying $T H S Z XSH$ on $A$ and then tracing out $B$.    
 
\end{enumerate}

\newcommand{\channel}{\mathcal{L}\left( \mathcal{H}_{2} \right)  \rightarrow \mathcal{L}\left( \mathcal{H}_{2} \right)} 

\newpage

\section{Diagonalize Noisy State.}
Let $\ket{A_{+}}$ and $\ket{A_{-}}$ be two orthogonal pure states over single qubit, i.e $\braket{A_{+} | A_{-}} = 0$ and let $\rho$ be the mixed state such that $\braket{A_{+}|\rho|A_{+}}  = 1 -p$ for some $p<1$. We would like to design a channel $\mathcal{E}_{A}: \channel $ that takes $\rho$ into:
    $$\rho^\prime = \mathcal{E}_{A}(\rho) = (1-p)\ket{A_{+}}\bra{A_{+}} + p\ket{A_{-}}\bra{A_{-}}$$ 
\begin{enumerate} 
    \item Solve it for the general case. (Hint: What does the orthogonality condition guarantee?). 
    \item \textbf{[ Bonus: ]} Let $\ket{A_1},\ket{A_2},\ket{A_3}, \ket{A_{4}}$ be four orthogonal states in $\mathcal{H}_{4}$. Suppose now that $\rho$ is supported on two qubits.  And for $i \in \{1,2,3,4\}$ denote by $p_{i}$ the probability to measure $\ket{A_i}$ given $\rho$. Namely $ \braket{A_i|\rho|A_i} = p_i$. Design a channel $\mathcal{E}_{A}$ that takes $\rho$ into:
    $$\rho^\prime = \mathcal{E}_{A}(\rho) =  \sum_{i=1}^{4}{p_{i}\ket{A_{i}}\bra{A_{i}}} $$  
\end{enumerate}

\section{Quantum Circuits.}
Prove equivalence for the following circuit pairs. 

\qCircEx{\textbf{[Not Mandatory]} Swapping Invariant of $CZ$:}{ $$ \begin{quantikz}
& \ghost{X} & \ctrl{1} & \\
&  \ghost{X} & \gate{Z} &
\end{quantikz}
=\begin{quantikz}
& \gate{Z}  & \ghost{X} & \\
& \ctrl{-1}  & \ghost{X} & 
\end{quantikz} $$}
{$X$,$Z$ - duality:}{$$ \begin{quantikz}
& \gate{H} & \gate{X} & \gate{H} & \\
& \gate{H} & \gate{Z} & \gate{H} &
\end{quantikz}
=\begin{quantikz}
&  \gate{Z} &  \\
&  \gate{X} &  
\end{quantikz} $$}
\qCircEx{Control $S$:}{ $$ \begin{quantikz}
& \gate{T} & \ctrl{1} &  &   \ctrl{1} &\\
&  \gate{T} & \targ{} & \gate{T^\dagger} & \targ{} & 
\end{quantikz}
=\begin{quantikz}
& \ctrl{1}  & \ghost{X} & \\
& \gate{S}  & \ghost{X} & 
\end{quantikz} $$}
{Reverse CNOT:}{$$ \begin{quantikz}
& \gate{H} & \ctrl{1} & \gate{H} & \\
& \gate{H} & \targ{} & \gate{H} &
\end{quantikz}
=\begin{quantikz}
&  \targ{}  & \ghost{X} \\
&  \ctrl{-1} & \ghost{X} 
\end{quantikz} $$}



\newpage

\newcommand{\abs}[1]{||#1||}
\section{[Not Mandatory] Measuring a Tensor Product.}

\noindent
Consider the task of distinguishing between the two quantum states 
\[
\ket{\Psi_0} = \ket{\psi_0}^{\otimes N} \quad \text{and} \quad \ket{\Psi_1} = \ket{\psi_1}^{\otimes N}
\]
Given with symmetric prior probabilities. The goal is to distinguish between them with minimal error probability. This question explores the structure of optimal measurements and error probabilities in this setting.

\begin{enumerate}
    \item Show that for $N = 1$, the Helstrom bound for the minimal error probability $p_e$ (achieved by the optimal measurement) satisfies:
    \[
    p_e(1 - p_e) = \frac{1}{4}||{\braket{\psi_0|\psi_1}}||^2.
    \] 
    
    \item Interpret the error probability $p_e$ as the posterior probability. That is, after one optimal measurement, the posterior distribution over the hypotheses is either $(p_e, 1 - p_e)$ or $(1 - p_e, p_e)$.

    \item Show that for $N = 1$ with a non-uniform prior $(q, 1 - q)$, the optimal error probability satisfies:
    \[
    p_e(1 - p_e) = q(1 - q)\abs{\braket{\psi_0|\psi_1}}^2.
    \]
    

    \item For $N = 2$, use parts (2) and (3) to show that performing the optimal measurement on the second copy—using the posterior from the first measurement as the prior—leads to a total error probability given by:
    \[
    p_e = \frac{1}{2}\left( 1 - \sqrt{1 - \abs{\braket{\psi_0|\psi_1}}^4} \right)
    \]

    \item Compute the inner product $\braket{\psi_0\psi_0 |\psi_1 \psi_1}$ in terms of $\braket{\psi_0|\psi_1}$. Conclude from this that the two-stage measurement strategy described above is optimal for $N = 2$.

    \item Extend the above reasoning by induction to any $N$, and describe the structure of the optimal measurement strategy.
\end{enumerate}
\printbibliography 
\end{document}

