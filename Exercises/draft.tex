
\documentclass[12pt,a4paper]{article}
\usepackage{framed}
\usepackage{advdate}
%-----------------------
% Begin document
%----------------------


\input{my-bib-file/usepackage.tex} 

\title{ \begin{framed} Quantum Information Theory - 67749 \\ 
Exercise 2, \today \end{framed}  }
\date{\vspace{-5ex}}
\addbibresource{my-bib-file/sample.bib} 

\begin{document}

\input{my-bib-file/newcommands.tex} 
\maketitle{ }    


\newcommand{\CCZ}{\textbf{CCZ}}
\newcommand{\CCX}{\textbf{CCX}}


\newcounter{enumcirc}
\setcounter{enumcirc}{1} 
\counterwithin{enumcirc}{section}


\newcommand{\advanceday}[1][11]{%
\begingroup
\AdvanceDate[#1]%
\today%
\endgroup
}%


\newcommand{\subqCircEx}[2]{\begin{subfigure}[t]{0.5\textwidth}
        \stepcounter{enumcirc} \caption*{ (\alph{enumcirc}) #1} \centering 
        #2
    \end{subfigure}
}

\newcommand{\qCircEx}[4]{\begin{figure*}[h!]
    \centering
    \subqCircEx{#1}{#2}
    ~ 
    \subqCircEx{#3}{#4}
\end{figure*}
}

\newcommand{\qCircExfullline}[2]{\begin{figure*}[h!]
    \stepcounter{enumcirc} \caption*{ (\alph{enumcirc}) #1}
        \centering 
        #2
\end{figure*}
}

\newcommand{\Tr}{\textbf{Tr}}

\section{Submission Guidelines.}
\begin{itemize}
    \item Due date - \advanceday. 
    \item Make sure your submission is clear. Unreadable assignments will get zero score.  
    \item Using any Generative AI tool is forbidden but not enforced. Yet, please keep in mind that we might call you for an interview about your assignment. 
\end{itemize}

\newpage


\section{Fidelity.}
Compute the fidelity between $\rho$ and $\sigma$ in the following cases: 
\begin{enumerate}
  \item Use the construction presented in the proof of Uhlman's to calculate the fidelity between: $\rho = \frac{4}{7}\ket{0}\bra{0} + \frac{3}{7}\ket{+}\bra{+}$  and $\sigma = \frac{3}{7}\ket{0}\bra{0} + \frac{4}{7}\ket{+}\bra{+}$. 
    

    \textbf{Solution.} Observes that $\sigma = H\rho H$, Since $H$ is a unitray then $\sigma^{\frac{1}{2}} = H \rho^{\frac{1}{2}} H$. Thus: 
    \begin{equation*}
      \begin{split}
        \max \Tr \sigma^{\frac{1}{2}} \rho^{\frac{1}{2}} V^{\dagger} & = \max  \Tr  \sigma^{\frac{1}{2}} H \sigma^{\frac{1}{2}} H V^{\dagger} = \Tr  \sigma^{\frac{1}{2}} H \sigma^{\frac{1}{2}} H \\ 
         %& =   \Tr  \sigma^{\frac{1}{2}} H \sigma^{\frac{1}{2}} = \Tr  \sigma H = \Tr \sigma
      \end{split}
    \end{equation*}


    \begin{equation*}
      \begin{split}
        \sigma & = \left( 1 - \frac{4}{7} + \frac{2}{7}  \right) \ket{0}\bra{0} + \frac{2}{7}\ket{0}\bra{1} +\frac{2}{7}\ket{1}\bra{0} +\frac{2}{7}\ket{1}\bra{1} = \begin{bmatrix}
             \frac{5}{7}  & \frac{2}{7}   \\
             \frac{2}{7} & \frac{2}{7}   
           \end {bmatrix} \\ 
            & \Rightarrow\left( \frac{5}{7} -  \lambda \right)\left(   \frac{2}{7} - \lambda \right) - \frac{4}{49} = 0 \\ 
           & \Rightarrow  \lambda^2 - \lambda  - \frac{6}{49} = 0 \Rightarrow  \lambda_{\pm} =  \frac{1}{2} \pm \frac{1}{2} \cdot \frac{5}{7}  \\ 
      \end{split}
    \end{equation*}
    So the eigenvalues are $ \lambda_{+} =  \frac{6}{7}$ and $\lambda_{-} = \frac{1}{7}$ with eigenvectors: 
    \begin{equation*}
      \begin{split}
        &  \ket{\omega_{+}} = \sqrt{\frac{2}{5}} \begin{bmatrix} 1 \\ \phantom{-} \frac{1}{2} \phantom{-}\end{bmatrix} \text{ and }  \ket{\omega_{-}} =\frac{1}{\sqrt{5}} \begin{bmatrix} 1 \\ -2 \phantom{-}\end{bmatrix} \\   
      \end{split}
    \end{equation*}
    Thus $\sigma^{\frac{1}{2}}$ equals: 


    \begin{equation*}
      \begin{split}
        \lambda_{+}^{\frac{1}{2}}\ket{\omega_{+}}\bra{\omega_{+}}+ \lambda_{-}^{\frac{1}{2}}\ket{\omega_{-}}\bra{\omega_{-}}= \sqrt{\frac{6}{7}}\frac{2}{5 } \begin{bmatrix} 1 & \frac{1}{2} \\ \frac{1}{2} & \frac{1}{4} \end{bmatrix} + \sqrt{\frac{1}{7}}  \frac{1}{5 } \begin{bmatrix} 1 & -2 \\ -2  & 4 \end{bmatrix} 
      \end{split}
    \end{equation*}

    \begin{equation*}
      \begin{split}
        \rho & = \left( 1 - \frac{3}{7} + \frac{3}{7 \cdot 2 }  \right) \ket{0}\bra{0} + \frac{3}{7 \cdot 2 }\ket{0}\bra{1} +\frac{3}{7 \cdot 2 }\ket{1}\bra{0} +\frac{3}{7 \cdot 2 }\ket{1}\bra{1} = \begin{bmatrix}
             \frac{11}{14}  & \frac{3}{14 }   \\
             \frac{3}{14 } & \frac{3}{14 }   
           \end {bmatrix} \\ 
            & \Rightarrow\left( \frac{11}{14} -  \lambda \right)\left(   \frac{3}{14} - \lambda \right) - \frac{9}{196} = 0 \\ 
            & \Rightarrow  \lambda^2 - \lambda  - \frac{9}{196} = 0 \Rightarrow  \lambda_{\pm} =  \frac{1}{2} \pm \frac{1}{2} \cdot \frac{4\sqrt{10}}{14}  = \frac{7 + 2\sqrt{10}}{14}, \frac{7 - 2\sqrt{10}}{14} \\ 
      \end{split}
    \end{equation*}
    W are almost ready to compute $\sigma^{\frac{1}{2}}H$ for that first let's compute: 
    \begin{equation*}
      \begin{split}
        \ket{\omega_{+}}\bra{\omega_{+}}H &= \frac{2}{5 \cdot \sqrt{2}} \begin{bmatrix} 1 & \frac{1}{2} \\ \frac{1}{2} & \frac{1}{4} \end{bmatrix} \begin{bmatrix} 1 & 1 \\ 1 & - 1 \end{bmatrix} = \frac{2 }{5 \cdot \sqrt{2}}\begin{bmatrix}  \frac{3}{2} & \frac{1}{2} \\  \frac{3}{4} & \frac{1}{4} \end{bmatrix} = \frac{1 }{5 \cdot \sqrt{2}}\begin{bmatrix}  3 & 1 \\  \frac{3}{2} & \frac{1}{2} \end{bmatrix}\\
        \ket{\omega_{-}}\bra{\omega_{-}}H &= \frac{1}{5 \cdot \sqrt{2} } \begin{bmatrix} 1 & -2 \\ -2  & 4 \end{bmatrix} \begin{bmatrix} 1 & 1 \\ 1 & - 1 \end{bmatrix} =\frac{1}{5 \cdot \sqrt{2}} \begin{bmatrix}  -1 & 3 \\  2 & -6 \end{bmatrix}
      \end{split}
    \end{equation*}
    Thus: 

    \begin{equation*}
      \begin{split}
      \sigma^{\frac{1}{2}}H = \frac{1}{5 \cdot \sqrt{2} }    \begin{bmatrix}  3 \cdot  \sqrt{\frac{6}{7}} - 1  \cdot  \sqrt{\frac{1}{7}} & 1  \cdot  \sqrt{\frac{6}{7}} + 4  \cdot  \sqrt{\frac{1}{7}}\\  \frac{3}{2}  \cdot  \sqrt{\frac{6}{7}} + 2 \cdot  \sqrt{\frac{1}{7}}  & \frac{1}{2}  \cdot  \sqrt{\frac{6}{7}} -6   \cdot  \sqrt{\frac{1}{7}}\end{bmatrix}      
      \end{split}
    \end{equation*}



    \begin{equation*}
      \begin{split}
          aaa
      \end{split}
    \end{equation*}

    Let $\ket{\psi_\rho}$ and $\ket{\psi_\sigma}$ be purifications of $\rho$ and $\sigma$.  
    \begin{equation*}
      \begin{split}
        \braket{\psi_{\rho}| \psi_{\sigma}} 
      \end{split}
    \end{equation*}
\end{enumerate}
\printbibliography 
\end{document}

