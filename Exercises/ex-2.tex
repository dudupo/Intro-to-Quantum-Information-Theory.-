\documentclass[12pt,a4paper]{article}
\usepackage{framed}
\usepackage{advdate}
%-----------------------
% Begin document
%----------------------


\input{my-bib-file/usepackage.tex} 

\title{ \begin{framed} Quantum Information Theory - 67749 \\ 
Exercise 2, \today \end{framed}  }
\date{\vspace{-5ex}}
\addbibresource{my-bib-file/sample.bib} 

\begin{document}

\input{my-bib-file/newcommands.tex} 
\maketitle{ }    


\newcommand{\CCZ}{\textbf{CCZ}}
\newcommand{\CCX}{\textbf{CCX}}


\newcounter{enumcirc}
\setcounter{enumcirc}{1} 
\counterwithin{enumcirc}{section}


\newcommand{\advanceday}[1][21]{%
\begingroup
\AdvanceDate[#1]%
\today%
\endgroup
}%


\newcommand{\subqCircEx}[2]{\begin{subfigure}[t]{0.5\textwidth}
        \stepcounter{enumcirc} \caption*{ (\alph{enumcirc}) #1} \centering 
        #2
    \end{subfigure}
}

\newcommand{\qCircEx}[4]{\begin{figure*}[h!]
    \centering
    \subqCircEx{#1}{#2}
    ~ 
    \subqCircEx{#3}{#4}
\end{figure*}
}

\newcommand{\qCircExfullline}[2]{\begin{figure*}[h!]
    \stepcounter{enumcirc} \caption*{ (\alph{enumcirc}) #1}
        \centering 
        #2
\end{figure*}
}

\newcommand{\Tr}{\mathbf{Tr}}

\section*{Submission Guidelines.}
\begin{itemize}
    \item Due date - \advanceday. 
    \item Make sure your submission is clear. Unreadable assignments will get zero score.  
    \item Using any Generative AI tool is forbidden but not enforced. Yet, please keep in mind that we might call you for an interview about your assignment. 
\end{itemize}

\newpage

%\section{AEP.}
%Recall the converse part of the classical AEP lemma\footnote{Taken from Prof. Ordentlich}:
%\begin{lemma}
  %Let $P_X$ be a pmf on $\mathcal{X}$ and let $\mathcal{B}^{(n)}$ be a set in $\mathcal{X}^n$ of size at most $2^{n\alpha}$. Then for any $\varepsilon > 0$ and $n$ large enough. 
  %\begin{equation*}
    %\begin{split}
      %\prb{X^n \in \mathcal{B}^{(n)}} = P^{\otimes n}_{X}\left( \mathcal{B}^{(n)} \right) \le \varepsilon + 2^{n(\alpha + \varepsilon - H(X))}
    %\end{split}
  %\end{equation*}
%\end{lemma}
%\begin{enumerate} 
  %\item Rewrite the classical AEP using the density matrix notation, namely point out each entity defined in the above statement and link it to its analogy in the QAEP lemma. [Hint\footnote{A good answer would be a single paragraph long.}]
%\end{enumerate}
%

\section{Entropy.}

\subsection{Information Quantties Properties. } 

\begin{enumerate}



  \item \textbf{Entropy Upper Bound.}

   \begin{enumerate}
     \item Show that the classical entropy is bounded from above by the logarithm of the dimension of the alphabet. Use the non-negativity of the divergence and its relation to the entropy.


    \item Similarly to the previews section, bound the von Neumann entropy using the non-negativity of the divergence.


  \end{enumerate}

  \item \textbf{Mutual Information Upper Bound.} 
     Prove that $S(A;B) \le 2\min \{S(A), S(B) \}$.




  \item \textbf{Mutual Information Chain Rule.} Prove the chain rule for the mutual infomration:
    \begin{equation*}
      \begin{split}
        S(AB;C) = S(A;B) + S(B;C|A) 
      \end{split}
    \end{equation*}


  \item \textbf{Entropy of Reduced Entangled State.} Consider the sate $\rho$ over the system $\mathcal{H}_{A} \otimes \mathcal{H}_{B}$.
    \begin{enumerate}
      \item Prove that if $S(A) \ge S(AB)$ then $\rho$ is entangled.   


      \item  Show, that the condition is not necessary, that is, give an example of an entangled state for which the reduced state $\rho_A$ has lower entropy than $\rho$. 


    \end{enumerate}


  \item \textbf{Conditional Mutual Information.} consider a joint system $ABX$ where $X$ is classical, show that the conditional mutual information $S(A;B|X)$ can be written as $\sum_x {p(x) S(A;B|X=x)}$, same as classical conditional MI.

    


  \item \textbf{Classical-Quantum States}. Recall the monotonicity of entropy with respect to revealing classical information.
     \begin{claim}
Let $X$ be a classical random variable, and let $\mathcal{H}_B$ be a subsystem for which, upon $X$'s value, the state $\rho_{x}$ is induced over $B$. Then: 
       \[ S(B) \leq S(X,B), \]    
    with equality if and only if the states $\rho_x$ are orthogonal, i.e., they have supports on orthogonal spaces.\end{claim}
In the lecture, we saw the equality when the quantum states induced on $B$ have orthogonal supports. Prove the inequality in the general case. Hint: Start by proving that if $f : \mathbb{R} \rightarrow \mathbb{R}$ is convex, then $A \mapsto \Tr f(A)$ is also convex.

\end{enumerate}

    

\subsection{Computing Entropy.}
Compute the entropy of the following densitiy matrices. 
\begin{enumerate}

  \item Let $V$ be a subset of $2^{n}$. Denote by $\ket{V}$ the uniform superposition over $V$, defined as $\ket{V} = \sum_{v \in V}{\ket{v}}$ (up to normalization). Denote by $\rho_{V}$ the uniform distribution over $V$, namely, sampling from $\rho_{V}$ gives any element $v \in V$ with equals probability. Compute the entropies of $\ket{V}$ and $\rho_{V}$.
    

  \item  Consider the fully entangled state $\ket{\Omega} = \sum_{x}{\ket{x,x}}$ (up to normalization) over the system $\mathcal{H}_{A} \otimes \mathcal{H}_{B}$. Compute the entropy $S(A,B)$ and the conditional entropy $S(A|B)$.
  

\end{enumerate}

\section{Fidelity.}
Compute the fidelity between $\rho$ and $\sigma$ in the following cases: 
\begin{enumerate}
  \item When $\rho$ and $\sigma$ commute. 



  \item When $\rho$ is mixed and $\sigma$ is pure.  


  \item When $\rho = \alpha I + \beta X $ and $\sigma = \gamma I + \delta Z$.
  \begin{remark}
Observes that the normalization condition $\Tr \rho = 1$ implies $\alpha = \gamma = 1$. Yet, for the sake of practice, we will keep the parametrization.
  \end{remark}

\end{enumerate}




%Let $\rho$ be the density matrix: $p \ket{\beta_{00}}\bra{\beta_{00}} + \frac{1}{3}(1-p)\sum_{i \neq 00}{\ket{\beta_{ij}}\bra{\beta_{ij}}}$.

\printbibliography 

\section{Schmedit.}
Prove that a pure state is entangled if and only if its Schmidt number is greater than one.



\end{document}

