%--------------------
% Packages
% -------------------
\documentclass[12pt,a4paper]{article}

\input{my-bib-file/usepackage.tex} 
%\usepackage[utf8x]{inputenc}
% \usepackage[T1]{fontenc}
%\usepackage{gentium}
% \usepackage{mathptmx} % Use Times Font


%\usepackage[pdftex]{graphicx} % Required for including pictures
%\usepackage[swedish]{babel} % Swedish translations
%\usepackage[pdftex,linkcolor=black,pdfborder={0 0 0}]{hyperref} % Format links for pdf
%\usepackage{calc} % To reset the counter in the document after title page
%\usepackage{enumitem} % Includes lists

%\frenchspacing % No double spacing between sentences
% \linespread{1.2} % Set linespace
%\usepackage[a4paper, lmargin=0.1666\paperwidth, rmargin=0.1666\paperwidth, tmargin=0.1111\paperheight, bmargin=0.1111\paperheight]{geometry} %margins
%\usepackage{parskip}

%\usepackage[all]{nowidow} % Tries to remove widows
%\usepackage[protrusion=true,expansion=true]{microtype} % Improves typography, load after fontpackage is selected
%
%\usepackage{lipsum} % Used for inserting dummy 'Lorem ipsum' text into the template
%\usepackage{amsmath}
%\usepackage{amsfonts}
%\usepackage{braket}
%\usepackage{subcaption}
%
%\usepackage{tikz}
%%\usepackage{chngcntr}
%\usetikzlibrary{quantikz2}
%
%\usepackage[
%backend=biber,
%style=alphabetic,
%sorting=ynt
%]{biblatex}
%\addbibresource{sample.bib}
%
%\usepackage{lmodern}
%-----------------------
% Set pdf information and add title, fill in the fields
%-----------------------
% \hypersetup{ 	
% pdfsubject = {},
% pdftitle = {},
% pdfauthor = {}
% }

\usepackage{framed}
%\usepackage{advdate}
%\usepackage[colorlinks=true]{hyperref}
%\usepackage{cleveref}
%\crefname{figure}{\textbf{Figure}}{\textbf{Figure}}
%-----------------------
% Begin document
%----------------------



\title{ \begin{framed} Quantum Information Theory - 67749 \\ 
Recitation 2, \today \end{framed}  }
%\date{\today}
\date{\vspace{-5ex}}



\begin{document}

\input{my-bib-file/sample.bib} 
\input{my-bib-file/newcommands.tex} 
%\begin{framed}
\maketitle{ }    
%\end{framed}



\newcommand{\CCZ}{\textbf{CCZ}}
\newcommand{\CCX}{\textbf{CCX}}


\newcounter{enumcirc}
\setcounter{enumcirc}{1} 
\counterwithin{enumcirc}{section}


\newcommand{\advanceday}[1][21]{%
\begingroup
\AdvanceDate[#1]%
\today%
\endgroup
}%


\newcommand{\subqCircEx}[2]{\begin{subfigure}[t]{0.5\textwidth}
        \stepcounter{enumcirc} \caption*{ (\alph{enumcirc}) #1} \centering 
        #2
    \end{subfigure}
}

\newcommand{\qCircEx}[4]{\begin{figure*}[h!]
    \centering
    \subqCircEx{#1}{#2}
    ~ 
    \subqCircEx{#3}{#4}
\end{figure*}
}

\newcommand{\qCircExfullline}[2]{\begin{figure*}[h!]
    \stepcounter{enumcirc} \caption*{ (\alph{enumcirc}) #1}
        \centering 
        #2
\end{figure*}
}

\section{Overview - Quantum States as Computational Resources.}
In the last lectures, we saw that quantum states can be considered as resources. In particular, we saw that shared $\textbf{EPR}$ pair ($\textbf{Bell}_{00}$) enables one:
\begin{enumerate}
    \item Transmit two classical bits by sending a single qubit, via the superdense-coding. 
    \item 'Teleoperate' a qubit by sending two classical bits. From an engineering point of view, it means that for having a complete quantum internet, it's enough to provide a mechanism to distribute $\textbf{EPR}$ pairs.
\end{enumerate}


\section{Dense Encoding.} ) 
\section{Quantum Teleportation.}

    \begin{figure}[h]
        \centering 
\begin{quantikz}
  \lstick{$\ket{\psi}$} &  &  &  \ctrl{1} & \gate{H} & \meter{} &  & \ctrl{2} & &  &  & &  & &  &\\
  \lstick{$\ket{0}$} & \gate{H} & \ctrl{1} & \targ{} &  & \meter{} &  \ctrl{1}  & & & & &  &  &&  &\\
  \lstick{$\ket{0}$} &  & \targ{} &   &  &  & \targ{} & \gate{Z} & &  &  & &  &&  &%\rstick{\ket{\psi}}
\end{quantikz}
     \caption{ Measuring the single-qubit state $\ket{\psi}$ at the $\{\ket{+}, \ket{-} \}$ base. }   
\end{figure}

\section{Gate Teleportation.}
Gate teleportation is a method to 'encode' operations by states. At the high level, given a precomputed state, it allows one to apply an operation (gate) by using (probably) simpler gates. The precomputed states are called \textbf{Magic States}.   
\subsection{Leading Example: $T$-Teleportation.}
Recall that the Clifford\footnote{Generated by $H, S$ and $CX$} + T is a universal quantum gate set. The Clifford group alone is considered from the computer science point of view a simple/weak computational class since it can be classically simulated \footnote{And conjectured to be strictly weaker than \textbf{P}}. Yet, we will see that given access to the magic $\ket{T} = T\ket{+}$, one can simulate the $T$ gate using only Clifford gates and measurements. 
    \begin{figure}[h]
        \centering 
        \begin{quantikz}
        \lstick{$\ket{\psi}$} & \ctrl{1}  & \gate{S} &  &&&&&&&&&&&&& \\
\lstick{$\ket{T}$} & \targ{} & \meter{} \wire[u][1]{c}    \\
        \end{quantikz}
     \caption{ Measuring the single-qubit state $\ket{\psi}$ at the $\{\ket{+}, \ket{-} \}$ base. }   
\label{fig:Hmeas}
\end{figure}

\begin{equation*}
    \begin{split}
        \left(\sum_{x}\alpha_{x}\ket{x}\right)\otimes\frac{1}{\sqrt{2}} \left(  \ket{0} + e^{i\frac{\pi}{4}}\ket{1}\right) & \overbrace{  \mapsto }^{ \textbf{CX} } \sum_{x,y}\frac{1}{\sqrt{2}}\alpha_{x}\ket{x}\ket{x\oplus y}e^{i\frac{\pi}{4} y} \\ 
        & \mapsto \ \ \begin{cases}
         \sum_{x}\alpha_x\ket{x}e^{i\frac{\pi}{4}x} = T\ket{\psi}   & \text{measured } 0 \\
           \sum_{x}\alpha_x\ket{x}e^{i\frac{\pi}{4}\bar{x}} & \text{measured } 1
        \end{cases} \\ 
        & \overbrace{  \mapsto }^{ \textbf{CS} } \begin{cases}
          T\ket{\psi}    \\
           \sum_{x}\alpha_x\ket{x}e^{i\left( \frac{\pi}{4}\bar{x} + \frac{\pi}{2}x \right)} =  \sum_{x}\alpha_x\ket{x}e^{i\frac{\pi}{4}}e^{i\left( \frac{\pi}{4}\bar{x} + \frac{\pi}{4}x \right)}
        \end{cases} \\ 
        & = \ \ \begin{cases}
          T\ket{\psi}    \\
           e^{i\frac{\pi}{4} } \sum_{x}\alpha_x\ket{x}e^{i\frac{\pi}{4}} = e^{i\frac{\pi}{4} } T\ket{\psi}
        \end{cases} 
    \end{split}
\end{equation*}

\subsection{Extends it.}

Let's extends it to a general gate. First create $\ket{\textbf{GHZ}_{2n}}$ state, then 

Let's split upon the measurement result. 
\begin{enumerate}
    \item If we measured $0$, means the states 'agreed' in the computational base. 
      \begin{equation*}
    \begin{split}
          \ket{\psi} \otimes \left( \sum_x\ket{x}\otimes U\ket{x} \right)
    \end{split}
\end{equation*}
\end{enumerate}

\section{Magic State Distillation.}

\textbf{Question.} Can we purify noisy magic states into high-fidelity ones, using only Clifford operations?

Magic state distillation is a procedure that uses many copies of noisy magic states, plus only Clifford gates and measurements, to produce fewer, higher-fidelity magic states.







\section{Uhlmann's theorem}

  \begin{equation*}
    \begin{split}
      \sum_{ij}{ \braket{ i;i | AB | j ;j } } &=  \sum_{ij}{ \braket{ i| A | j} \braket{i | B | j } } =  \sum_{ij}{ \braket{ i| A | j} \braket{j | B^\top | i } } = \sum_{i}{ \braket{i | A B^\top | i } } = \mathbf{Tr} AB^\top
    \end{split}
  \end{equation*}



\begin{equation*}
  \begin{split}
  \ket{\psi_\rho} = \sum_{i}{ \left( \rho^{\frac{1}{2}}\ket{\psi_i} \right) \ket{i} } \\ 
  \end{split}
\end{equation*}

\begin{equation*}
  \begin{split}
  \ket{\psi_\sigma} = \sum_{i}{ \left( \sigma^{\frac{1}{2}}\ket{\psi^\prime_i} \right) \ket{i^\prime} } =  
  \sum_{i}{ \left( \sigma^{\frac{1}{2}} U_{1} \ket{\psi_{i}} \right)  U_{2}\ket{i} }   
  \end{split}
\end{equation*}



\begin{equation*}
  \begin{split}
    \max | \braket{ \psi_\rho | \psi_\sigma } |^2  & =
  \max | \sum_{i}{ \left( \bra{\psi_i} \rho^{\frac{1}{2}}\right) \bra{i}  \left( \sigma^{\frac{1}{2}} U_{1} \ket{\psi_{j}} \right)  U_{2}\ket{j} } |^{2} \\
  & = \max |   \sum_{i}{  \mathbf{Tr} \left[  \left( \bra{\psi_i} \rho^{\frac{1}{2}}\right) \bra{i}  \left( \sigma^{\frac{1}{2}} U_{1} \ket{\psi_{j}} \right)  U_{2}\ket{j} \right] |^{2}}  \\
  & = \max |   \sum_{i}{  \mathbf{Tr} \left[  \left( \sigma^{\frac{1}{2}} U_{1} \ket{\psi_{j}}\bra{\psi_i} \rho^{\frac{1}{2}}\right)   \left( U_{2}\ket{j} \bra{i}\right) \right]  }  |^{2} \\
  & = \max |   \sum_{i}{  \mathbf{Tr} \left[  \left(  \sigma^{\frac{1}{2}}\rho^{\frac{1}{2}}U_{1} \ket{\psi_{j}}\bra{\psi_i} \right)   \left( U_{2}\ket{j} \bra{i}\right) \right]  }  |^{2} \\
  & = \max \left|     \mathbf{Tr} \left[  \left(  \sigma^{\frac{1}{2}}\rho^{\frac{1}{2}}U_{1}  \right) \otimes U_{2}  \right]    \right|^{2} \\
  & \le \left|   \mathbf{Tr} \sqrt{      \rho^{\frac{1}{2}} \sigma^{\frac{1}{2}} \sigma^{\frac{1}{2}}\rho^{\frac{1}{2}}   }   \right|^{2} = \left|   \mathbf{Tr} \sqrt{      \rho^{\frac{1}{2}} \sigma \rho^{\frac{1}{2}}   }   \right|^{2}  \\
  \end{split}
\end{equation*}

Notice that $\ket{i}\bra{j}$ is unitray since. 



\printbibliography 

\end{document}

