\documentclass[12pt,a4paper]{article}
\usepackage{framed}
\usepackage{advdate}
%-----------------------
% Begin document
%----------------------


\input{my-bib-file/usepackage.tex} 

\title{ \begin{framed} Quantum Information Theory - 67749 \\ 
Guided Exercise on Recitation, \today \end{framed}  }
\date{\vspace{-5ex}}
\addbibresource{my-bib-file/sample.bib} 

\begin{document}

\input{my-bib-file/newcommands.tex} 
\maketitle{ }    


\newcommand{\CCZ}{\textbf{CCZ}}
\newcommand{\CCX}{\textbf{CCX}}


\newcounter{enumcirc}
\setcounter{enumcirc}{1} 
\counterwithin{enumcirc}{section}


\newcommand{\advanceday}[1][21]{%
\begingroup
\AdvanceDate[#1]%
\today%
\endgroup
}%


\newcommand{\subqCircEx}[2]{\begin{subfigure}[t]{0.5\textwidth}
        \stepcounter{enumcirc} \caption*{ (\alph{enumcirc}) #1} \centering 
        #2
    \end{subfigure}
}

\newcommand{\qCircEx}[4]{\begin{figure*}[h!]
    \centering
    \subqCircEx{#1}{#2}
    ~ 
    \subqCircEx{#3}{#4}
\end{figure*}
}

\newcommand{\qCircExfullline}[2]{\begin{figure*}[h!]
    \stepcounter{enumcirc} \caption*{ (\alph{enumcirc}) #1}
        \centering 
        #2
\end{figure*}
}

\newcommand{\Tr}{\mathbf{Tr}}

\section{CSS codes.}

\begin{enumerate}

  \item Consider the CSS code $Q(C_{X}, C_{Z})$ and let $f \in \mathbb{F}_{2}^{n}$. Denote by $Z^{f}$ the operator which on the $i$th qubit acts trivially if $f_{i} = 0$ and otherwise applies $Z$. 

    Assume that $|f| < d(C_{Z})/2$. Explain how to correct the noisy state $Z^{f}\ket{C_{Z}^\perp}$ and convince yourself that you can also correct a noisy state of the form $\left( Z^{f_{1}}X^{f_{2}} + Z^{f_{3}} \right) \ket{C_{Z}^\perp}$.

\inb{ We will apply the Hadamard gate $H^{\otimes n}$, apply the classical decoder for $C_{Z}$, and eventually we will apply the Hadamard again to return to the original form. The state is evolving as follows:
      \begin{equation*}
       \begin{split}
         Z^{f}\ket{C_{Z}^\perp} & \overbrace{\rightarrow}^{ H^{n}} X^{f}\ket{C_{Z}} \overbrace{\rightarrow}^{ \text{Decoding}  } \ket{C_{Z}} \\
            & \overbrace{\rightarrow}^{ H^{n}} \ket{C_{Z}^\perp}
        \end{split}
      \end{equation*}
    } 

  \item Now suppose that we want to correct the logical codeword $\ket{x + C_{Z}^\perp}$. How to correct it?   

    \inb { We will repeat over the same precdure as above, and notice that:
      \begin{equation*}
        \begin{split}
          Z^{f}\ket{x + C_{Z}^\perp} &=  Z^{f}X^{x}\ket{ C_{Z}^\perp} \overbrace{\rightarrow}^{ H^{n}} H^{n}Z^{f}X^{x}\ket{C_{Z}} \\ 
          & = (-1)^{f \cdot x} Z^{x} \ket{C_{Z} + f}\overbrace{\rightarrow}^{ \text{Decoding}  } X^{f}  (-1)^{f \cdot x} Z^{x} \ket{C_{Z} + f}\\
         & =   Z^{x} X^{f} \ket{C_{Z} + f} \ket{C_{Z}} = Z^{x} \ket{C_{Z} +f}\\
            & \overbrace{\rightarrow}^{ H^{n}} \ket{x + C_{Z}^\perp}
        \end{split}
      \end{equation*}<++>

    }

  \item  Prove that the relation $C_{X} \subset C_{Z}^{\perp}$ implies $H_{Z}H_{X}^\top = 0$, where $H_{Z}$ and $H_{X}$ are the parity check matrices of the codes $C_{X}, C_{Z}$.

    \inb{ \textbf{[Solution.]} $H_{X}^\top$ is the generator matrix of the subspace spanned by its columns (True for any matrix), namely by $H_{X}$ rows, which, by definition, are all the vectors perpendicular to codewords in $C_X$. Thus, $H_{X}^\top$ is the generator matrix for the code $C_{X}^\perp$. Since $C_{X}^\perp \subset C_{Z}$, we get the relation $H_{Z}H_{X}^\top = 0$. 
    }
  \item Prove that it cannot hold that both of the codes $C_{X}, C_{Z}$ are LDPC codes with non-constant distance, and that they compose a CSS code.

    \inb{ \textbf{[Solution.]} By the relation $H_{Z}H_{X}^\top = 0$, we have that any check of $H_{X}$ is a codeword of $C_{Z}$, so requiring that $C_{X}$ is an LDPC code implies that $C_{Z}$ has codewords at weight $O(1)$.}


  \item Take a minute to think about the result above. Try to understand why your TA believes that this should be the entry point to explain why the world that we see around us is classical and not quantum.



\end{enumerate}

\end{document}

