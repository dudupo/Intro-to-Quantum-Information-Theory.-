\documentclass{beamer}
%\usepackage{split}

\usepackage{amsmath}
\usepackage{amsfonts}
\usepackage{braket}
\usepackage{subcaption}
\usepackage{tikz}
\usepackage{quantikz}
\begin{document} 

\newcommand*{\Tr}{\textbf{Tr }}


\begin{frame}
    \title{Final Recitation – Information Theory, Application for Quantum Fault Tolerance.}
    \author{David Ponarovsky}
    \date{\today}
    \titlepage
\end{frame}


\begin{frame}

\frametitle{Introduction}
\begin{block}{Today:}
  \begin{itemize}[<+->]
  \item Nosiy circuit and noisy computation. 
  \item Classical no-go for reversible noisy computation at logarithmic depth (and polynomial space).
  \item Quantum case. 
  \item Trading (local) entropy for space.
\end{itemize}
\end{block}
\end{frame}

\begin{frame}
  \frametitle{Nosiy Circuit.}

\begin{quantikz}[row sep=0.3cm, column sep=0.7cm]
\lstick{$q_1$} & \gate[wires=8]{U_0} & \gate{\mathcal{N}} & \gate[wires=8]{U_1}   & \gate{\mathcal{N}} & \gate[wires=8]{U_2} & \gate{\mathcal{N}}& \qw \\
\lstick{$q_2$} &                      & \gate{\mathcal{N}} &                      & \gate{\mathcal{N}} &                     & \gate{\mathcal{N}} & \qw \\
\lstick{$q_3$} &                      & \gate{\mathcal{N}} &                      & \gate{\mathcal{N}} &                     & \gate{\mathcal{N}} & \qw \\
\lstick{$q_4$} &                      & \gate{\mathcal{N}} &                      & \gate{\mathcal{N}} &                     & \gate{\mathcal{N}} & \qw \\
\lstick{$q_5$} &                      & \gate{\mathcal{N}} &                      & \gate{\mathcal{N}} &                     & \gate{\mathcal{N}} & \qw \\
\lstick{$q_6$} &                      & \gate{\mathcal{N}} &                      & \gate{\mathcal{N}} &                     & \gate{\mathcal{N}} & \qw \\
\lstick{$q_7$} &                      & \gate{\mathcal{N}} &                      & \gate{\mathcal{N}} &                     & \gate{\mathcal{N}} & \qw \\
\lstick{$q_8$} &                      & \gate{\mathcal{N}} &                      & \gate{\mathcal{N}} &                     & \gate{\mathcal{N}} & \qw
\end{quantikz}
\end{frame}

\begin{frame}{Nosiy Circuit.}
  \uncover<+->{
  \begin{definition}{ $p$- Depolarizing Channel. } 
    The qubit depolarizing channel with parameter $ p \in [0,1] $ is the quantum channel $ \mathcal{D}_p $ defined by:
\begin{equation*}
  \begin{split}
\mathcal{D}_p(\rho) = (1 - p) \rho + p \cdot \frac{I}{2}
  \end{split}
\end{equation*}

where $ \rho $ is a single-qubit density matrix and $ I $ is the identity matrix.

  \end{definition}
}
\uncover<+->{
  \begin{definition}{$p$-Noisy Circuit.}
    Given a circuit $C$ (regardless of the model), its $p$-noisy version $\tilde{C}$ is the circuit obtained by alternately taking layers from $C$ and then passing each (qu)bit through a $p$-Depolarizing channel.
  \end{definition}
}
\end{frame}

\begin{frame}
  \frametitle{Classical no-go.}
  \begin{block}{Claim-Informal.}
    We can't compute when subjected to noise, 
    \uncover<+->{ after a while, the bits become garbage.}
  \end{block}

\uncover<+->{  
\begin{block}{Claim}
  Denote b $X = (X_{1}, X_{2}, .., X_{m})$ and $Y = (Y_1, Y_2 , .. ,Y_m)$ the input and the output distrubtions of reversible $p$-noisy compuation at widith $m$ (bits) and depth $d$. Then: 
  \uncover<+->{
  \begin{equation*}
    \begin{split}
      m - H(Y) \le \gamma^{d}\left( m - H(X) \right) 
    \end{split}
  \end{equation*}}
  \uncover<+->{In particular, for $d = \Omega(\log m)$ we have $H(Y) \rightarrow m$.}
\end{block}
}
\end{frame}
\begin{frame}
  \frametitle{Classical no-go.}
  
\begin{block}{Claim}
  Let $Y$ be a bit given by moving $X$ trough BSC($p$), Then there is $\gamma_{p} < 1$ such :
  \begin{equation*}
    \begin{split}
      1 - H(Y) \le \gamma \left( 1 - H(X) \right)
    \end{split}
  \end{equation*}
\end{block}

\end{frame}

\begin{frame}
  \frametitle{Classical no-go.}
  Denote by $\delta$ the parameter for which $X$ distributed as $\sim \text{Bin}(\frac{1+\delta}{2})$. First notice that:  
  \begin{equation*}
    \begin{split}
      \textbf{Pr}\left( Y = 1 \right) = \frac{1+\delta}{2} (1 -p) + \frac{1-\delta}{2}p = \frac{1 + \delta - 2\delta p}{2}  
    \end{split}
  \end{equation*}
  So $Y \sim \text{Bin}(  \frac{1 -\delta( 1 -  2 p)}{2} )$, Or $\delta \mapsto 1-2p \delta$. 
\end{frame}

\begin{frame}
  \frametitle{Classical no-go.}
Now expand $ 1 - H(X)$ to it's Taylor Seryias at $\delta$ gives: 
\begin{equation*}
  \begin{split}
    1 - H(X) &\uncover<+->{= 1 - \frac{1}{2} \left( \left( 1 + \delta \right) \log       \left( \frac{1 + \delta}{2} \right) + \left( 1 - \delta \right) \log       \left( \frac{1 - \delta}{2} \right) \right) \\ }
    &\uncover<+->{=- \frac{1}{2} \left( \left( 1 + \delta \right) \log       \left( \frac{1 + \delta}{2} \right) + \left( 1 - \delta \right) \log       \left( \frac{1 - \delta}{2} \right) \right) \\ }
    &\uncover<+->{= - \frac{1}{2} \ \cdot \  (1+ \delta) \sum_{i=1}^{\infty}{ \frac{(-1)^{n+1}\delta^{n}}{n} } +  (1- \delta) \sum_{i=1}^{\infty}{ \frac{(-1)^{n+1}(-\delta)^{n}}{n} }\\}
    &\uncover<+->{=  - \frac{1}{2} \ \cdot \   \sum_{i=1}^{\infty}{2 \frac{\delta^{2n}}{2n} }  - \sum_{i=1}^{\infty}{ 2\frac{\delta^{2n}}{2n-1} }\\ }
    &\uncover<+->{= \sum_{ i =1 }^{\infty} \frac{\delta^{2n} }{ 2n(2n-1)  } }
  \end{split}
\end{equation*}

\uncover<+->{Denote the above by $K(\delta)$}

\end{frame}


\begin{frame}
  \frametitle{Classical no-go.}
  Now, observes that: 
  \begin{equation*}
    \begin{split}
      1 - H(Y) &\uncover<+->{= K(2p \delta) = \sum_{ i =1 }^{\infty} \frac{(2 p\delta)^{2n} }{ 2n(2n-1)  }\\}
      &\uncover<+->{\le ( 1 -2p)^{2}K(\delta) = ( 1 - 2p)^{2}( 1 - H(X) )}
    \end{split}
  \end{equation*}
  \uncover<+->{And notice that since $p < 1$ we have $\gamma < 1$,}\uncover<+->{ noitce also that inequlity is symmetric to $p \mapsto 1 - p$, in paritcular the entropy is not increase if eithrr $p =0$ or $p=1$.}
\end{frame}  


\begin{frame}
  \frametitle{Classical no-go.}
  
\begin{block}{Claim}
  Let $Y = \left(Y_{1}, Y_{2}, .., Y_{m}\right)$ be a bit given by moving each of $X_i \in$ $X = \left(X_{1}, X_{2}, .., X_{m}\right)$  trough BSC($p$). Then:
\uncover<+->{
  \begin{equation*}
    \begin{split}
      m - H(Y) \le \gamma\left( m - H(X) \right)
    \end{split}
  \end{equation*}}
\end{block}

\end{frame}
\begin{frame}
  \frametitle{Classical no-go.}
  \begin{block}{Proof.} 
  \begin{equation*}
    \begin{split}
      m - H\left( Y_{1}, Y_{2}, .., Y_{m} \right) &\uncover<+->{= m - \sum_{i} H\left( Y_{i} | Y_{1}, Y_{2}, .., Y_{i-1} \right) \\ }
      &\uncover<+->{ \le m - \sum_{i} H\left( Y_{i} | X_{1}, X_{2}, .., X_{i-1} \right) \\ }
      &\uncover<+->{ \le \sum_{i} 1 - H\left( Y_{i} | X_{1}, X_{2}, .., X_{i-1} \right) \\ }
      &\uncover<+->{ \le \sum_{i} \gamma\left(1 - H\left( X_{i} | X_{1}, X_{2}, .., X_{i-1} \right)\right) \\ }
      &\uncover<+->{ \le \gamma \sum_{i} \left(1 - H\left( X_{i} | X_{1}, X_{2}, .., X_{i-1} \right)\right)  \\}
      &\uncover<+->{= \gamma\left( m - H\left( X \right) \right)}
    \end{split}
  \end{equation*}
\end{block}
\end{frame}
\begin{frame}
  \frametitle{Classical no-go.}
  
  
\begin{block}{Claim}
  Denote b $X = (X_{1}, X_{2}, .., X_{m})$ and $Y = (Y_1, Y_2 , .. ,Y_m)$ the input and the output distrubtions of reversible $p$-noisy compuation at widith $m$ (bits) and depth $d$. Then: 
  \uncover<+->{
  \begin{equation*}
    \begin{split}
      m - H(Y) \le \gamma^{d}\left( m - H(X) \right) 
    \end{split}
  \end{equation*}}
  \uncover<+->{In particular, for $d = \Omega(\log m)$ we have $H(Y) \rightarrow m$.}


\end{block}
\end{frame}
\begin{frame}
  \frametitle{Quantum no-go.}
  Let's continue to the quantum case. \uncover<+->{From now on we will denote by $I(\rho) = \dim \rho - S(\rho)$.}\uncover<+->{For repeating the above w'll have to prove more claims.}
\end{frame}
\begin{frame}
  \frametitle{Quantum no-go.}
  
\begin{block}{Claim}
  Let $\rho_{1}$ be a reduce density matrix of $\rho$ Then:  
  \uncover<+->{
  \begin{equation*}
    \begin{split}
      - \Tr \rho \log \left( \rho_{1} \otimes I \right)= S(\rho_{1})
    \end{split}
  \end{equation*}
}
\end{block}
\end{frame}
\begin{frame}
  \frametitle{Quantum no-go.}
  First consider the case in which $\rho$ is a tensor of $\rho_{1}$ namely $\rho = \rho_{1} \otimes \rho_{2}$, Then clearly $\rho$ and $\log \rho_{1} \otimes I$ commute. 
  Denote by $\lambda_{1},..\lambda_{n}$ and $\mu_{1},..\mu_{m}$ the eigen values of $\rho_{1}$ and $\rho_2$. So the trace equals: 
  \begin{equation*}
    \begin{split}
      \sum \lambda_{i}\mu_{j}\log(\lambda_{i} \cdot 1) &\uncover<+->{= \left( \sum \mu_{j} \right) \left( \sum_i \lambda_{i} \log \lambda_{i} \right) \\}
      &\uncover<+->{= \left( \Tr \rho_{2} \right) \sum_i \lambda_{i} \log \lambda_{i} = -S(\rho_{1})}
    \end{split}
  \end{equation*}
\end{frame}

\begin{frame}
  \frametitle{Quantum no-go.}
  Let's use the notation $\sum_{A_k}\rho|_{A_k}$ to denote the sum over all the reduced matrices over $k$ qubits.  
\begin{block}{Claim}
  Let $\rho$ be a density matrix over $n$ qubits then:  
  \uncover<+->{
  \begin{equation*}
    \begin{split}
      { n \choose k }^{-1}\sum_{A_k} I(\rho|_{A_k}) \le \frac{k}{n}I(\rho)
    \end{split}
  \end{equation*}}
\end{block}
\end{frame}
\begin{frame}
  \frametitle{Quantum no-go.}
  \begin{block}{Proof}
Let $\rho_{1}$ be $\rho^{\otimes k { n \choose k } }$ and let $\rho_{2} = \left( \prod_{A_{k}} \rho |_{A_{k}} \right)^{\otimes n}$.  
  \begin{equation*}
    \begin{split}
      0 \ge S(\rho_{2} | \rho_{1} ) &\uncover<+->{= \Tr \left( \rho_{1} \left( \log \rho_{1} - \log \rho_{2} \right) \right) = -S(\rho_{1}) - \Tr \left( \rho_{1} \log \rho_{2} \right) \\ }
      &\uncover<+->{= -k { n \choose k } S(\rho) - \sum_{A_k} \Tr \left( \rho_{1} \log \left( \rho|_{A_{k}} \right)^n \otimes I^n \right)}
    \end{split}
  \end{equation*}
  \uncover<+->{Now observes that $ \rho|_{A_{K}}^{\otimes n} $ is a reduced density matrix of $\rho_{1}$. So we get:

  \begin{equation*}
    \begin{split}
      0 &\uncover<+->{ \le  -k { n \choose k } S(\rho) - \sum_{A_k} n S\left( \rho |_{A_{k}} \right) \\ }
      \Rightarrow  &\uncover<+->{ \sum_{A_{k}}I\left( \rho|_{A_{k}} \right)\le \frac{k}{n}{ n \choose k } I\left( \rho \right)}
    \end{split}
  \end{equation*}

  }
\end{block}
\end{frame}

\begin{frame}
  \frametitle{Quantum no-go.}
\begin{block}{Claim}
  Let $\rho$ be a density matrix of $n$ qubits. Let each qubit be replaced with independent probability $p$ by a fully mixed qubit denoted by $\upsilon$, to give the density matrix $\sigma$. Then $I\left( \sigma \right) \le \left( 1 - p  \right) I \left( \rho \right)$. 
\end{block}
\end{frame}

\begin{frame}
  \frametitle{Quantum no-go.}
Let us write: 
\begin{equation*}
  \begin{split}
    \sigma = \sum^{n}_{k=1}\sum_{A_{k}}p^{n-k}\left( 1 - p  \right)^{k}\rho|_{A_{k}}\otimes \upsilon^{n-k}
  \end{split}
\end{equation*}
\uncover<+->{
By the concavity of the entropy (convexity of $I$), We have:
\begin{equation*}
  \begin{split}
    I\left(\sigma\right) &\uncover<+->{\le \sum^{n}_{k=1}\sum_{A_{k}}p^{n-k}\left( 1 - p  \right)^{k}\left[ I\left(\rho|_{A_{k}}\right) + (n-k) I(\upsilon) \right] \\ }
    &\uncover<+->{= \sum^{n}_{k=1}\sum_{A_{k}}p^{n-k}\left( 1 - p  \right)^{k} I\left(\rho|_{A_{k}}\right)  \\ }
    &\uncover<+->{ \le \sum^{n}_{k=1}\sum_{A_{k}}p^{n-k}\left( 1 - p  \right)^{k} \frac{k}{n} { n \choose k } I\left(\rho\right)  \\ }
    &\uncover<+->{ = \left( 1 - p \right)I\left(\rho\right)}
  \end{split}
\end{equation*}}
\end{frame}

\begin{frame}
  \frametitle{Entropy-Space tradeoff.}

Zeros Distillation. Consider the unitary majority gate over $3$ qubits, which on the computational basis sets the last $3$rd bit to be the majority, and acts as follows:
  
  \begin{equation*}
    \begin{split}
      M \ket{0,0,1} & \mapsto \ket{1,1,0} \\ 
      M \ket{1,1,0} & \mapsto \ket{0,0,1} \\ 
    \end{split}
  \end{equation*} 
And acts trivially on every other configuration.

\end{frame}



\begin{frame}
  \frametitle{Entropy-Space tradeoff.}
  the probability of the thried qubit to be $1$ is still less than $1$. 
  There is $p_{0}$ such that for any $p < p_{0}$, it follows that if each qubit has a probability less than $p$ of being at $\ket{1}$, then after a cycle of Noise $\rightarrow$ Majority, the probability of the third qubit being $\ket{1}$ is still less than $p$.


\end{frame}
\begin{frame}
  \frametitle{Entropy-Space tradeoff.}
Proof. After the noise round, the density matrices of each qubit:
  \begin{equation*}
    \begin{split}
      &\uncover<+->{\le \left( 1- p \right)\left(  \left( 1 -p \right) \ket{0}\bra{0} + p\ket{1}\bra{1}  \right) + p \frac{1}{2}\left( \ket{0}\bra{0} + \ket{1}\bra{1} \right) \\ }
      &\uncover<+->{= \left( \left( 1-p \right)^{2} + \frac{1}{2}p \right)\ket{0}\bra{0} + \left( \left( 1-p \right)p + \frac{1}{2}p \right)\ket{1}\bra{1} \\}
      &\uncover<+->{= " \ket{0}\bra{0} + \left( 3/2 p - p^{2} \right) \ket{1}\bra{1} }
    \end{split}
  \end{equation*}
\uncover<+->{
And for convenience, let's bound by looking at the following noisier state, where each qubit is in:
  \begin{equation*}
    \begin{split}
      \left( 1 - 2p \right)\ket{0}\bra{0} + 2p \ket{1}\bra{1}
    \end{split}
  \end{equation*}
}
\uncover<+->{Now, after applying the majority gate, the kets whose third bit is $1$ are kets in $\rho^{\otimes 3}$ that absorb at least two flips.}\uncover<+->{ Thus, after tracing out the first two qubits, the coefficient of $\ket{1}\bra{1}$ would be less than:

  \begin{equation*}
    \begin{split}
      \left( 2p \right)^{3} + 3\left( 1 - 2p \right) \left( 2p \right)^2 \uncover<+->{ \le \left( 2p \right)^{3} + 3\left( 2p \right)^2 }
    \end{split}
\end{equation*}}
\uncover<+->{And for $p_0 = \frac{1}{6}$ we have that this probability is less than $p$. }

\end{frame}


\begin{frame}
    \title{The End.}
    \titlepage
\end{frame}


\begin{frame}

\end{frame}

\begin{frame}

    \title{Why Tracing-out the first and second qubit doesn't work?.}
    \titlepage
\end{frame}

\begin{frame}
  \frametitle{Entropy-Space tradeoff.}


\begin{block}{Claim} Consider the noisy zero $\rho = \left( 1 - \frac{p}{2} \right) \ket{0}\bra{0} + \frac{p}{2}\ket{1}\bra{1}$. Then: 
  \begin{equation*}
    \begin{split}
      \Tr_{[1,2]} M \rho^{3} = \left( 1 - \frac{3}{4}p^2 + \frac{1}{4}p^{3} \right) \ket{0}\bra{0} + ,,
    \end{split}
  \end{equation*}
\end{block}
\end{frame}
\begin{frame}
  \frametitle{Entropy-Space tradeoff.}

  
  \begin{equation*}
    \begin{split}
      \rho^{\otimes 3} &\uncover<+->{= \left( 1 - \frac{p}{2} \right)^3 \ket{000}\bra{000} \\ }
      &\uncover<+->{+ \left( 1 - \frac{p}{2} \right)^{2} \frac{p}{2}\left( \ket{001}\bra{001} + \ket{010}\bra{010} + \ket{100}\bra{100}  \right) \\}
      &\uncover<+->{+ \left( 1 - \frac{p}{2} \right)\left(\frac{p}{2}\right)^2 \left( \ket{110}\bra{110} + \ket{101}\bra{101} + \ket{011}\bra{011} \right) \\ }
      &\uncover<+->{+ \left( \frac{p}{2} \right)^{3} \ket{111}\bra{111}\\}
      M\rho^{\otimes 3} &\uncover<+->{= " \\}
      &\uncover<+->{+ \left( 1 - \frac{p}{2} \right)^{2} \frac{p}{2}\left( \ket{110}\bra{110} + "  \right) \\}
      &\uncover<+->{+ \left( 1 - \frac{p}{2} \right)\left(\frac{p}{2}\right)^2 \left( \ket{001}\bra{001} + " \right) \\ }
      &\uncover<+->{+ " }
    \end{split}
  \end{equation*}
\end{frame}
\begin{frame}
  \frametitle{Entropy-Space tradeoff.}  
  \begin{equation*}
    \begin{split}
      \Tr_{[1,2]}M\rho^{\otimes 3} &\uncover<+->{= \left( \left( 1 - \frac{p}{2} \right)^3 + 3 \left( 1 - \frac{p}{2} \right)^{2} \frac{p}{2} \right) \ket{0}\bra{0} +  \\ }
      &\uncover<+->{+ \left( 3\left( 1 - \frac{p}{2} \right)\left( \frac{p}{2} \right)^2 + \left( \frac{p}{2} \right)^{3} \right) \ket{1}\bra{1}\\}
      &\uncover<+->{= \left( 1 - \frac{3}{4}p^2 + \frac{1}{4}p^{3} \right) \ket{0}\bra{0} + ,,}
    \end{split}
  \end{equation*}
\end{frame}

\begin{frame}{Entropy-Space tradeoff.}
    \begin{figure}
        \centering
        \includegraphics[width=\textwidth]{/srv/shared/graph.png}
    \end{figure}
\end{frame}

\begin{frame}
  \frametitle{Entropy-Space tradeoff.}

  \begin{equation*}
    \begin{split}
      &\uncover<+->{ \left( 1 - \frac{3}{4}p^2 + \frac{1}{4}p^{3} \right) \ket{0}\bra{0} + ,, \\}
      \mapsto &\uncover<+->{ \left( 1 - p\right) \left( 1 - \frac{3}{4}p^2 + \frac{1}{4}p^{3} \right) \ket{0}\bra{0} + \frac{p}{2} \ket{0}\bra{0} \\ }
      = &\uncover<+->{ 1 - \frac{1}{2}p - \frac{3}{4}p^{2} + p^3 - \frac{1}{4}p^{4}}
    \end{split}
  \end{equation*}
\end{frame}

\begin{frame}{Your Title Here}

    \begin{figure}
        \centering
        \includegraphics[width=\textwidth]{/srv/shared/graph_after_noise.png}
    \end{figure}
\end{frame}

\begin{frame}{Your Title Here}

\end{frame}

\begin{frame}{Your Title Here}

\end{frame}
\end{document}
