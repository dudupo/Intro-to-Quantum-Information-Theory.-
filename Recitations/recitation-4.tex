\documentclass{beamer}
%\usepackage{split}

\usepackage{amsmath}
\usepackage{amsfonts}
\usepackage{braket}
\usepackage{subcaption}
\usepackage{tikz}
\usepackage{svg}
\begin{document} 

\newcommand*{\Tr}{\textbf{Tr }}


\begin{frame}
    \title{Final Recitation – Information Theory, Application for Quantum Fault Tolerance.}
    \author{David Ponarovsky}
    \date{\today}
    \titlepage
\end{frame}


\begin{frame}

\frametitle{Introduction}
\begin{itemize}
    \item Brief overview of the topic
    \item Importance and relevance
    \item Objectives of the presentation
\end{itemize}
\end{frame}

\begin{frame}
\frametitle{Key Points}
\begin{itemize}
    \item Main point 1
    \item Main point 2
    \item Main point 3
\end{itemize}
\end{frame}

\begin{frame}
  \frametitle{Your Title Here}
  
\begin{block}{Claim}
  Let $Y$ be a bit given by moving $X$ trough BSC($p$), Then there is $\gamma_{p} < 1$ such :
  \begin{equation*}
    \begin{split}
      1 - H(Y) \le \gamma \left( 1 - H(X) \right)
    \end{split}
  \end{equation*}
\end{block}

\end{frame}

\begin{frame}
  \frametitle{Your Title Here}
  Denote by $\delta$ the parameter for which $X$ distributed as $\sim \text{Bin}(\frac{1+\delta}{2})$. First notice that:  
  \begin{equation*}
    \begin{split}
      \textbf{Pr}\left( Y = 1 \right) = \frac{1+\delta}{2} (1 -p) + \frac{1-\delta}{2}p = \frac{1 + \delta - 2\delta p}{2}  
    \end{split}
  \end{equation*}
  So $Y \sim \text{Bin}(  \frac{1 -\delta( 1 -  2 p)}{2} )$, Or $\delta \mapsto 1-2p \delta$. 
\end{frame}

\begin{frame}
  \frametitle{Your Title Here}
Now expand $ 1 - H(X)$ to it's Taylor Seryias at $\delta$ gives: 
\begin{equation*}
  \begin{split}
  1 - H(X) &= 1 - \frac{1}{2} \left( \left( 1 + \delta \right) \log       \left( \frac{1 + \delta}{2} \right) + \left( 1 - \delta \right) \log       \left( \frac{1 - \delta}{2} \right) \right) \\ 
  &=- \frac{1}{2} \left( \left( 1 + \delta \right) \log       \left( \frac{1 + \delta}{2} \right) + \left( 1 - \delta \right) \log       \left( \frac{1 - \delta}{2} \right) \right) \\ 
  &= - \frac{1}{2} \ \cdot \  (1+ \delta) \sum_{i=1}^{\infty}{ \frac{(-1)^{n+1}\delta^{n}}{n} } +  (1- \delta) \sum_{i=1}^{\infty}{ \frac{(-1)^{n+1}(-\delta)^{n}}{n} }\\
&=  - \frac{1}{2} \ \cdot \   \sum_{i=1}^{\infty}{2 \frac{\delta^{2n}}{2n} }  - \sum_{i=1}^{\infty}{ 2\frac{\delta^{2n}}{2n-1} }\\ 
      &= \sum_{ i =1 }^{\infty} \frac{\delta^{2n} }{ 2n(2n-1)  } 
  \end{split}
\end{equation*}

Denote the above by $K(\delta)$

\end{frame}


\begin{frame}
  \frametitle{Your Title Here}
  Now, observes that: 
  \begin{equation*}
    \begin{split}
      1 - H(Y) &= K(2p \delta) = \sum_{ i =1 }^{\infty} \frac{(2 p\delta)^{2n} }{ 2n(2n-1)  }\\
      &\le ( 1 -2p)^{2}K(\delta) = ( 1 - 2p)^{2}( 1 - H(X) )
    \end{split}
  \end{equation*}
  And notice that since $p < 1$ we have $\gamma < 1$, noitce also that inequlity is symmetric to $p \mapsto 1 - p$, in paritcular the entropy is not increase if eithrr $p =0$ or $p=1$.
\end{frame}  


\begin{frame}
  \frametitle{Your Title Here}
  
\begin{block}{Claim}
  Let $Y = \left(Y_{1}, Y_{2}, .., Y_{m}\right)$ be a bit given by moving each of $X_i \in$ $X = \left(X_{1}, X_{2}, .., X_{m}\right)$  trough BSC($p$). Then:

  \begin{equation*}
    \begin{split}
      m - H(Y) \le \gamma\left( m - H(X) \right)
    \end{split}
  \end{equation*}
\end{block}

\end{frame}
\begin{frame}
  \frametitle{Your Title Here}
  
  \begin{equation*}
    \begin{split}
      m - H\left( Y_{1}, Y_{2}, .., Y_{m} \right) &= m - \sum_{i} H\left( Y_{i} | Y_{1}, Y_{2}, .., Y_{i-1} \right) \\ 
      & \le m - \sum_{i} H\left( Y_{i} | X_{1}, X_{2}, .., X_{i-1} \right) \\ 
      & \le \sum_{i} 1 - H\left( Y_{i} | X_{1}, X_{2}, .., X_{i-1} \right) \\ 
      & \le \sum_{i} \gamma\left(1 - H\left( X_{i} | X_{1}, X_{2}, .., X_{i-1} \right)\right) \\ 
      & \le \gamma \sum_{i} \left(1 - H\left( X_{i} | X_{1}, X_{2}, .., X_{i-1} \right)\right)  \\
      &= \gamma\left( m - H\left( X \right) \right)
    \end{split}
  \end{equation*}
\end{frame}
\begin{frame}
  \frametitle{Your Title Here}
  
  
\begin{block}{Claim}
  Denote b $X = (X_{1}, X_{2}, .., X_{m})$ and $Y = (Y_1, Y_2 , .. ,Y_m)$ the input and the output distrubtions of reversible $p$-noisy compuation at widith $m$ (bits) and depth $d$. Then, there: 
  \begin{equation*}
    \begin{split}
      m - H(Y) \le \gamma^{d}\left( m - H(X) \right) 
    \end{split}
  \end{equation*}
  In particular, for $d = \Omega(\log m)$ we have $H(Y) \rightarrow m$. 


\end{block}


  % Content goes here
\end{frame}
\begin{frame}
  \frametitle{Your Title Here}
  % Content goes here
\end{frame}
\begin{frame}
  \frametitle{Your Title Here}
  
\begin{block}{Claim}
  Let $\rho_{1}$ be a reduce density matrix of $\rho$ Then:  
  \begin{equation*}
    \begin{split}
      - \Tr \rho \log \left( \rho_{1} \otimes I \right)= S(\rho_{1})
    \end{split}
  \end{equation*}
\end{block}
\end{frame}
\begin{frame}
  \frametitle{Your Title Here}
  First consider the case in which $\rho$ is a tensor of $\rho_{1}$ namely $\rho = \rho_{1} \otimes \rho_{2}$, Then clearly $\rho$ and $\log \rho_{1} \otimes I$ commute. 
  Denote by $\lambda_{1},..\lambda_{n}$ and $\mu_{1},..\mu_{m}$ the eigen values of $\rho_{1}$ and $\rho_2$. So the trace equals: 
  \begin{equation*}
    \begin{split}
      \sum \lambda_{i}\mu_{j}\log(\lambda_{i} \cdot 1) &= \left( \sum \mu_{j} \right) \left( \sum_i \lambda_{i} \log \lambda_{i} \right) \\
      &= \left( \Tr \rho_{2} \right) \sum_i \lambda_{i} \log \lambda_{i} = -S(\rho_{1})
    \end{split}
  \end{equation*}
\end{frame}

\begin{frame}
  \frametitle{Your Title Here}
  Let's use the notation $\sum_{A_k}\rho|_{A_k}$ to denote the sum over all the reduced matrices over $k$ qubits.  
\begin{block}{Claim}
  Let $\rho$ be a density matrix over $n$ qubits then:  
  \begin{equation*}
    \begin{split}
      { n \choose k }^{-1}\sum_{A_k} I(\rho|_{A_k}) \le \frac{k}{n}I(\rho)
    \end{split}
  \end{equation*}
\end{block}
\end{frame}
\begin{frame}
  \frametitle{Your Title Here}
Let $\rho_{1}$ be $\rho^{\otimes k { n \choose k } }$ and let $\rho_{2} = \left( \prod_{A_{k}} \rho |_{A_{k}} \right)^{\otimes n}$.  
  \begin{equation*}
    \begin{split}
      0 \ge S(\rho_{2} | \rho_{1} ) &= \Tr \left( \rho_{1} \left( \log \rho_{1} - \log \rho_{2} \right) \right) = -S(\rho_{1}) - \Tr \left( \rho_{1} \log \rho_{2} \right) \\ 
      &= -k { n \choose k } S(\rho) - \sum_{A_k} \Tr \left( \rho_{1} \log \left( \rho|_{A_{k}} \right)^n \otimes I^n \right)
    \end{split}
  \end{equation*}
  Now observes that $ \rho|_{A_{K}}^{\otimes n} $ is a reduced density matrix of $\rho_{1}$. So we get: 

  \begin{equation*}
    \begin{split}
      0 & \le  -k { n \choose k } S(\rho) - \sum_{A_k} n S\left( \rho |_{A_{k}} \right) \\ 
      \Rightarrow  & \sum_{A_{k}}I\left( \rho|_{A_{k}} \right)\le \frac{k}{n}{ n \choose k } I\left( \rho \right)
    \end{split}
  \end{equation*}
\end{frame}

\begin{frame}
  \frametitle{Your Title Here}
\begin{block}{Claim}
  Let $\rho$ be a density matrix of $n$ qubits. Let each qubit be replaced with independent probability $p$ by a fully mixed qubit denoted by $\upsilon$, to give the density matrix $\sigma$. Then $I\left( \sigma \right) \le \left( 1 - p  \right) I \left( \rho \right)$. 
\end{block}
\end{frame}

\begin{frame}
  \frametitle{Your Title Here}
Let us write: 
\begin{equation*}
  \begin{split}
    \sigma = \sum^{n}_{k=1}\sum_{A_{k}}p^{n-k}\left( 1 - p  \right)^{k}\rho|_{A_{k}}\otimes \upsilon^{n-k}
  \end{split}
\end{equation*}
By the concavity of the entropy (convexity of $I$), We have:
\begin{equation*}
  \begin{split}
    I\left(\sigma\right) &\le \sum^{n}_{k=1}\sum_{A_{k}}p^{n-k}\left( 1 - p  \right)^{k}\left[ I\left(\rho|_{A_{k}}\right) + (n-k) I(\upsilon) \right] \\ 
    &= \sum^{n}_{k=1}\sum_{A_{k}}p^{n-k}\left( 1 - p  \right)^{k} I\left(\rho|_{A_{k}}\right)  \\ 
    & \le \sum^{n}_{k=1}\sum_{A_{k}}p^{n-k}\left( 1 - p  \right)^{k} \frac{k}{n} { n \choose k } I\left(\rho\right)  \\ 
    & = \left( 1 - p \right)I\left(\rho\right)
  \end{split}
\end{equation*}
\end{frame}

\begin{frame}
  \frametitle{Your Title Here}

Zeros Distillation. Consider the unitary majority gate over $3$ qubits, which on the computational basis sets the last $3$rd bit to be the majority, and acts as follows:
  
  \begin{equation*}
    \begin{split}
      M \ket{0,0,1} & \mapsto \ket{1,1,0} \\ 
      M \ket{1,1,0} & \mapsto \ket{0,0,1} \\ 
    \end{split}
  \end{equation*} 
And acts trivially on every other configuration.

\begin{block}{Claim} Consider the noisy zero $\rho = \left( 1 - \frac{p}{2} \right) \ket{0}\bra{0} + \frac{p}{2}\ket{1}\bra{1}$. Then: 
  \begin{equation*}
    \begin{split}
      \Tr_{[1,2]} M \rho^{3} = \left( 1 - \frac{p}{2} - \frac{p^2}{4} + \frac{p^3}{8} \right) \ket{0}\bra{0} + ,,
    \end{split}
  \end{equation*}
\end{block}
\end{frame}
\begin{frame}
  \frametitle{Your Title Here}

  
  \begin{equation*}
    \begin{split}
      \rho^{\otimes 3} &= \left( 1 - \frac{p}{2} \right)^3 \ket{000}\bra{000} \\ 
      &+ \left( 1 - \frac{p}{2} \right)^{2} \frac{p}{2}\left( \ket{001}\bra{001} + \ket{010}\bra{010} + \ket{100}\bra{100}  \right) \\
      &+ \left( 1 - \frac{p}{2} \right)\left(\frac{p}{2}\right)^2 \left( \ket{110}\bra{110} + \ket{101}\bra{101} + \ket{011}\bra{011} \right) \\ 
      &+ \left( \frac{p}{2} \right)^{3} \ket{111}\bra{111}\\
      M\rho^{\otimes 3} &= " \\
      &+ \left( 1 - \frac{p}{2} \right)^{2} \frac{p}{2}\left( \ket{110}\bra{110} + "  \right) \\
      &+ \left( 1 - \frac{p}{2} \right)\left(\frac{p}{2}\right)^2 \left( \ket{001}\bra{001} + " \right) \\ 
      &+ " 
    \end{split}
  \end{equation*}
\end{frame}
\begin{frame}
  \frametitle{Your Title Here}  
  \begin{equation*}
    \begin{split}
      \Tr_{[1,2]}M\rho^{\otimes 3} &= \left( \left( 1 - \frac{p}{2} \right)^3 + 3 \left( 1 - \frac{p}{2} \right)^{2} \frac{p}{2} \right) \ket{0}\bra{0} +  \\ 
      &+ \left( 3\left( 1 - \frac{p}{2} \right)\left( \frac{p}{2} \right)^2 + \left( \frac{p}{2} \right)^{3} \right) \ket{1}\bra{1}\\
    \end{split}
  \end{equation*}
\end{frame}


\begin{frame}{Your Title Here}
    \begin{figure}
        \centering
        \includesvg[width=\textwidth]{/srv/shared/graph.svg}
        \caption{Description of the SVG figure}
    \end{figure}
\end{frame}

\end{document}
