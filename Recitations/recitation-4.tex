\documentclass{beamer}
%\usepackage{split}

\usepackage{amsmath}
\usepackage{amsfonts}
\usepackage{braket}
\usepackage{subcaption}
\usepackage{tikz}
\begin{document} 
 

\begin{frame}
    \title{Final Recitation – Information Theory, Application.}
    \author{David Ponarovsky}
    \date{\today}
    \titlepage
\end{frame}


\begin{frame}

\frametitle{Introduction}
\begin{itemize}
    \item Brief overview of the topic
    \item Importance and relevance
    \item Objectives of the presentation
\end{itemize}
\end{frame}

\begin{frame}
\frametitle{Key Points}
\begin{itemize}
    \item Main point 1
    \item Main point 2
    \item Main point 3
\end{itemize}
\end{frame}

\begin{frame}
  \frametitle{Your Title Here}
  
\begin{block}{Claim}
  Let $Y$ be a bit given by moving $X$ trough BSC($p$). Then:
  \begin{equation*}
    \begin{split}
      1 - H(Y) \le \left( 1 - p^{2} \right)\left( 1 - H(X) \right)
    \end{split}
  \end{equation*}
\end{block}

\end{frame}

\begin{frame}
  \frametitle{Your Title Here}
  Denote by $\delta$ the parameter for which $X$ distributed as $\sim \text{Bin}(\frac{1+\delta}{2})$. First notice that:  
  \begin{equation*}
    \begin{split}
      \textbf{Pr}\left( Y = 1 \right) = \frac{1+\delta}{2} (1 -p) + \frac{1-\delta}{2}p = \frac{1 - 2\delta p}{2}  
    \end{split}
  \end{equation*}
  So $Y \sim \text{Bin}(  \frac{1 - 2\delta p}{2} )$, Or $\delta \mapsto -2p \delta$. 
\end{frame}

\begin{frame}
  \frametitle{Your Title Here}
Now expand $ 1 - H(X)$ to it's Taylor Seryias at $\delta$ gives: 
\begin{equation*}
  \begin{split}
  1 - H(X) &= 1 - \frac{1}{2} \left( \left( 1 + \delta \right) \log       \left( \frac{1 + \delta}{2} \right) + \left( 1 - \delta \right) \log       \left( \frac{1 - \delta}{2} \right) \right) \\ 
  &=- \frac{1}{2} \left( \left( 1 + \delta \right) \log       \left( \frac{1 + \delta}{2} \right) + \left( 1 - \delta \right) \log       \left( \frac{1 - \delta}{2} \right) \right) \\ 
  %&= - \frac{1}{2} \ \cdot \  (1+ \delta) \sum_{i=1}^{\infty}{ \frac{(-1)^{n+1}\delta^{n}{n} }} +  (1- \delta) \sum_{i=1}^{\infty}{ \frac{(-1)^{n+1}(-\delta)^{n}{n} }}\\
      %&=  - \frac{1}{2} \ \cdot \   \sum_{i=1}^{\infty}{2 \frac{\delta^{2n}{2n} }}  - \sum_{i=1}^{\infty}{ 2\frac{\delta^{2n}{2n-1} }}\\ 
      %&= \sum_{ i =1 }^{\infty} \frac{\delta^{2n} }{ 2n(2n-1)  } 
  \end{split}
\end{equation*}

Denote the above by $K(\delta)$

\end{frame}


\begin{frame}
  \frametitle{Your Title Here}
  Now, observes that: 
  \begin{equation*}
    \begin{split}
      1 - H(Y) = K(2p \delta) &= \sum_{ i =1 }^{\infty} \frac{(2 p\delta)^{2n} }{ 2n(2n-1)  }\\
      \le 2p^{2}K(\delta) = 2p^{2}( 1 - H(X) )
    \end{split}
  \end{equation*}
\end{frame}  


\begin{frame}
  \frametitle{Your Title Here}
  
\begin{block}{Claim}
  Let $Y = \left(Y_{1}, Y_{2}, .., Y_{m}\right)$ be a bit given by moving each of $X_i \in$ $X = \left(X_{1}, X_{2}, .., X_{m}\right)$  trough BSC($p$). Then:

  \begin{equation*}
    \begin{split}
      m - H(Y) \le \left( 1 - p^{2} \right)\left( m - H(X) \right)
    \end{split}
  \end{equation*}
\end{block}

\end{frame}
\begin{frame}
  \frametitle{Your Title Here}
  
  \begin{equation*}
    \begin{split}
      m - H\left( Y_{1}, Y_{2}, .., Y_{m} \right) &= m - \sum_{i} H\left( Y_{i} | Y_{1}, Y_{2}, .., Y_{i-1} \right) \\ 
      & \le m - \sum_{i} H\left( Y_{i} | X_{1}, X_{2}, .., X_{i-1} \right) \\ 
      & \le \sum_{i} 1 - H\left( Y_{i} | X_{1}, X_{2}, .., X_{i-1} \right) \\ 
      & \le \sum_{i} \left( 1 - p^{2} \right)\left(1 - H\left( X_{i} | X_{1}, X_{2}, .., X_{i-1} \right)\right) \\ 
      & \le \left( 1 - p^{2} \right) \sum_{i} \left(1 - H\left( X_{i} | X_{1}, X_{2}, .., X_{i-1} \right)\right)  \\
      &= \left( 1 - p^{2} \right) \left( m - H\left( X \right) \right)
    \end{split}
  \end{equation*}
\end{frame}
\begin{frame}
  \frametitle{Your Title Here}
  % Content goes here
\end{frame}
\begin{frame}
  \frametitle{Your Title Here}
  % Content goes here
\end{frame}
\begin{frame}
  \frametitle{Your Title Here}
  % Content goes here
\end{frame}
\end{document}
